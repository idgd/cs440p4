\documentclass[11pt]{article}

\usepackage{setspace}
\usepackage[activate={true,nocompatibility},
            final,
            tracking=true,
            kerning=true,
            spacing=true,
            factor=1100,
            stretch=10,
            shrink=10]{microtype}
\usepackage{listings}
\usepackage[dvipsnames]{xcolor}

\lstset{language=Python,
        basicstyle=\footnotesize \ttfamily,
        commentstyle=\color{Green},
        keywordstyle=\color{Cyan},
        stringstyle=\color{Magenta},
        showstringspaces=false,
        numbers=left,
        numbersep=5pt,
        numberstyle=\tiny,
        title=\lstname,
				tabsize=4}

\microtypecontext{spacing=nonfrench}

\doublespacing

\setlength\parskip{1ex}
\setlength\parindent{1em}

\title{Project 4}
\author{Isaac Dudney}
\date{\today}

\begin{document}

\maketitle

\section{Psuedocode}

A general description of how my code works is simple.
Each of the cache replacement algorithms is represented by a Pure Function which takes two lists as input and returns a single list.
The two list inputs are the pattern (eg \texttt{ABCDEABC}) and the available slots (which are initialized to None and replaced as the algorithm fills them).
Each function initializes a special list of strings which store the text output, which is returned at the end of the function.
Every function iterates over the pattern character by character and performs comparisons on the slots to determine behavior.
It then appends output text to the list of strings according to the behavior; for example, if it gets a cache hit, it appends a \texttt{+} to the appropriate slot.

The main thread starts by taking user input; it exits if the input isn't integers, or if the input is out of bounds.
It then initializes each necessary item (pattern and slots) using the input integers.
The functions are then defined, along with a utility function to properly print and wrap the output.
Each algorithm is then called and printed using the utility function, after which the program exits.

However, I will also write psuedocode for each algorithm, including the printing utility function, here in a set of listings.
Starting with \texttt{FIFO}.

\subsection{FIFO}

\begin{lstlisting}
def FIFO(pattern,slots):
	# index in slots to replace
	index = 0
	# return special printing struct
	# inits with "FIFO  #: " for number of slots + 1
	# where # is the slot number
	r = ["FIFO  " + i + ": " for i in range(len(slots) + 1)]
	# redefine first line to have the input pattern
	# with 1 space after every char as per assignment
	r[0] = "Ref Str: " + " ".join(pattern)

	# for each item in the pattern, do stuff
	for f in pattern:

		# if the item is in the slots
		# wri():
		# magic write to proper place function for psuedocode
		# it writes a "+ " to the printing struct when it hits
		# and the proper output (f + " ") if it misses
		if f in slots: wri("hit")

		# if there's an empty slot
		elif None in slots:
			# magic write
			wri(f)
			# insert at first available empty slot
			slots[slots.index(None)] = f

		# if no empty slots exist
		else:
			wri(f)
			# replace slot at index
			slots[index] = f
			# if index is smaller than number of slots, increment
			if index < len(slots) - 1:
				index++
			# otherwise, reset to 0
			else:
				index = 0
	# return print struct
	return(r)

\end{lstlisting}

As stated in the comments, \texttt{wri()} is a magic function purely for psuedocode; looking in the actual code, each instance is replaced by three lines that depend on the algorithm for their logic.
In the end, it ends up doing the same thing, producing good output, as can be seen here:

\begin{lstlisting}
Ref Str: D C D A B D B C C D
FIFO  1: D   + A   D       +
FIFO  2:   C     B   + C +
\end{lstlisting}

The output will also wrap nicely, but we will cover that portion more extensively in the output section. For now, we will take a look at \texttt{LRU}.

\subsection{LRU}

\begin{lstlisting}
def LRU(pattern,slots):
	# here, instead of a plain index, we keep a time index of each use
	# the time is relative to the place in pattern
	# EG in the pattern ABCDEABC, the first B is time 1 (0-indexed)
	index = [0 for f in slots]

	# same special printing struct
	r = ["LRU   " + i + ": " for i in range(len(slots) + 1)]
	r[0] = "Ref Str: " + " ".join(pattern)

	# for each item in the pattern, do stuff, by index!
	for f in range(len(pattern)):

		# if the item is in the slots
		if pattern[f] in slots:
			# same magic write
			wri("hit")
			# this time, set the index of the hit to the current time
			# f, or time, here is the index as defined in line 4
			index[slots.index(pattern[f])] = f
			# breaking it down: slots.index() finds the corresponding
			# slot for the index list. We pass in the current item
			# in the pattern (pattern[f]) to find where we hit
			# we then set that to the current time in the index

		# if there's an empty slot
		elif None in slots:
			# magic write
			wri(pattern[f])
			# insert at first available empty slot
			# this time, record the index for reuse in next line
			i = slots.index(None)
			slots[i] = pattern[f]
			index[i] = f

		# if no empty slots exist
		else:
			# lots of indexes here! to put it in english:
			# find the number index of the smallest item in the index var
			i = index.index(min(index))
			# magic write
			wri(pattern[f])
			# set least recently used slot to new item
			slots[i] = pattern[f]
			# set current slot to most recently used
			index[i] = f
	# return print struct
	return(r)

\end{lstlisting}

The output of this function with the same input as the previously shown \texttt{FIFO} output would look like so:
\vspace{-2em}

\begin{lstlisting}
Ref Str: D C D A B D B C C D
LRU   1: D   +   B   +     D
LRU   2:   C   A   D   C +
\end{lstlisting}

Moving on to the \texttt{MIN} algorithm.

\subsection{MIN}

\begin{lstlisting}
def MIN(pattern,slots):
	# despite it being the technically impossible one,
	# it actually has the simplest implementation, given the pattern

	# printing struct
	r = ["MIN   " + i + ": " for i in range(len(slots) + 1)]
	r[0] = "Ref Str: " + " ".join(pattern)

	# for each item in pattern by index
	for f in range(len(pattern)):

		# if item exists, cache hit
		if pattern[f] in slots: wri("hit")

		# if empty slot exists
		elif None in slots:
			wri(pattern[f])
			slots[slots.index(None)] = pattern[f]

		# if no empty slots exist
		else:
			# here's where the logic happens!
			index_list = []
			# iterate over the slots
			# find the next index (by pattern) for each item
			for g in slots:
				# if it exists, append its next index
				# to the index_list
				if g in pattern[f:]:
					index_list.append(pattern[f:].index(g) + f)
				# otherwise append an impossible index
				# (as len(pattern) !> 100)
				else:
					index_list.append(101)
			# set index in slot to the index
			# of the greatest item in index_list
			if 101 in index_list: i = index_list.index(101)
			else: i = slots.index(pattern[max(index_list)])
			# write cache miss
			wri(pattern[f])
			# set slot to missed item
			slots[i] = pattern[f]
	# return print struct
	return(r)

\end{lstlisting}

Hopefully the explanations in the last \texttt{else} statement made sense.
They are an accurate description of what occurs.
The output of this function, with the same input as the others, would look like so:

\begin{lstlisting}
Ref Str: D C D A B D B C C D
MIN   1: D   +     +       +
MIN   2:   C   A B   + C +
\end{lstlisting}

Now, finally, the easiest one to describe, \texttt{RAND}.

\subsection{RAND}

\begin{lstlisting}
def RAND(pattern,slots):
	# printing struct
	r = ["RAND  " + i + ": " for i in range(len(slots) + 1)]
	r[0] = "Ref Str: " + " ".join(pattern)

	# for each in pattern
	for f in pattern:
		# if cache hit
		if f in slots: wri("hit")
		# if empty slot exists
		if None in slots:
			wri(f)
			slots[slots.index(None)] = f
		# if no empty slots available
		else:
			# pick a random one
			slots[random(len(slots))] = f
			wri(f)

	return(r)
\end{lstlisting}

As the \texttt{RAND} function is non-deterministic, here is the output of one iteration of the same input as the others.
However, most outputs of \texttt{RAND} should look generally similar.

\begin{lstlisting}
Ref Str: D C D A B D B C C D
RAND  1: D   +     +   C + D
RAND  2:   C   A B   +
\end{lstlisting}

\subsection{Main}

Finally for the psuedocode, we have to go over the main code; I will indicate the above function definitions with a simple \texttt{def NAME(...): ...} to indicate the function has been shrunk for size concerns.
Here is the main body of the code in psuedocode:

\begin{lstlisting}
# take user input
pattern_length = input("pattern length: ")
unique_characters = input("unique characters: ")
number_of_slots = input("number of slots: ")

# if any errors occur
if any not int: exit("not numbers")
if 10 > pattern_length > 101: exit("out of range")
if 2 > unique_characters,number_of_slots > 11: exit("out of range")

# initialize the pattern
pattern = [choose(lowercase_ascii[:unique_characters])
           for f in range(pattern_length)]
# initialize the slots
slots = [None for f in range(number_of_slots)]

def FIFO(...): ...
def LRU(...): ...
def MIN(...): ...
def RAND(...): ...

# special printing function, will show detail in output section
dep pr(...): ...

pr(FIFO(pattern,slots))
pr(LRU(pattern,slots))
pr(MIN(pattern,slots))
pr(RAND(pattern,slots))

\end{lstlisting}

\section{Output}

Here, I will discuss the the printing function for both unwrapped (exactly to spec) and wrapped (not to spec, but hopefully in good taste).
Firstly, I will show output of the program as a whole for an unwrapped print.

\begin{lstlisting}
ptrn lngth: 25
uniq chars: 5
slot lngth: 3

Ref Str: B D D E A A C A D E A B C C B C D C D E C A B D E
FIFO  1: B       A +   +   E     C +   +   +     + A     E
FIFO  2:   D +       C       A           D   +       B
FIFO  3:       E         D     B     +         E       D
---------------------------------------------------------
Ref Str: B D D E A A C A D E A B C C B C D C D E C A B D E
LRU   1: B       A +   +     +           D   +     A     E
LRU   2:   D +       C     E     C +   +   +     +     D
LRU   3:       E         D     B     +         E     B
---------------------------------------------------------
Ref Str: B D D E A A C A D E A B C C B C D C D E C A B D E
MIN   1: B       A +   +     + B     +   D   +         +
MIN   2:   D +           + E                   +         +
MIN   3:       E     C           + +   +   +     + A B
---------------------------------------------------------
Ref Str: B D D E A A C A D E A B C C B C D C D E C A B D E
RAND  1: B                   A   C +   + D C D     A   D E
RAND  2:   D +       C   D     B     +               +
RAND  3:       E A +   +   E                   + C
\end{lstlisting}

As can be seen, each piece of output is according to spec in the assignment sheet.
The user's input is accepted and each function takes the appropriate variables to produce correct output.
Each function (except \texttt{RAND}, due to the random nature, and \texttt{pr}, as it prints to the output) is also Pure.
This means it edits no global state, only working on its own copies of input and producing unique output (not modifying global state and returning that).

Now that we have gone over the output in its spec-compliant form, now let's delve into the output in wrapping form.
Here is the output if it has to wrap:

\begin{lstlisting}
ptrn lngth: 70
uniq chars: 4
slot lngth: 2

Ref Str: C B C B D C B D D C D B D D D A B A D A B A A B A B B
FIFO  1: C   +   D   B     C     D + +   B     A   + +   +
FIFO  2:   B   +   C   D +   + B       A   + D   B     +   + +
######## wrapping 1
Ref Str: A D A B D D B B B C D C C B B D D D D B C C C A D B C
FIFO  1:   D     + +       C   + + B +         +       A   B
FIFO  2: +   + B     + + +   D         + + + +   C + +   D   C
######## wrapping 2
Ref Str:  B C B D C B B
FIFO  1:    C     + B +
FIFO  2:  B   + D
--------------------------------------------------------------
--------------------------------------------------------------
-----------------------
Ref Str: C B C B D C B D D C D B D D D A B A D A B A A B A B B
LRU   1: C   +   D   B     C   B       A   +   +   + +   +
LRU   2:   B   +   C   D +   +   + + +   B   D   B     +   + +
######## wrapping 1
Ref Str: A D A B D D B B B C D C C B B D D D D B C C C A D B C
LRU   1:   D   B     + + +   D     B +         +       A   B
LRU   2: +   +   D +       C   + +     D + + +   C + +   D   C
######## wrapping 2
Ref Str:  B C B D C B B
LRU   1:    +   D   B +
LRU   2:  B   +   C
--------------------------------------------------------------
--------------------------------------------------------------
-----------------------
Ref Str: C B C B D C B D D C D B D D D A B A D A B A A B A B B
MIN   1: C   +     + B     C   B         +   D   B     +   + +
MIN   2:   B   + D     + +   +   + + + A   +   +   + +   +
######## wrapping 1
Ref Str: A D A B D D B B B C D C C B B D D D D B C C C A D B C
MIN   1: +   + B     + + + C   + + B +         + C + + A     C
MIN   2:   +     + +         +         + + + +           + B
######## wrapping 2
Ref Str:  B C B D C B B
MIN   1:    +     + B +
MIN   2:  B   + D
--------------------------------------------------------------
--------------------------------------------------------------
-----------------------
Ref Str: C B C B D C B D D C D B D D D A B A D A B A A B A B B
RAND  1: C   +   D C   D +   + B D + +   B     A   + +   +
RAND  2:   B   +     +     C           A   + D   B     +   + +
######## wrapping 1
Ref Str: A D A B D D B B B C D C C B B D D D D B C C C A D B C
RAND  1:   +   B     + + +     C + B +         +       A
RAND  2: +   +   D +       C D         + + + +   C + +   D B C
######## wrapping 2
Ref Str:  B C B D C B B
RAND  1:  +   + D
RAND  2:    +     + B +

\end{lstlisting}

As can be seen, each one wraps around all of its lines to produce a generally nice-looking output with clear indications of what is related to what.

\end{document}
